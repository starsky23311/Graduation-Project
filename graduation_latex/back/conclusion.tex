% !Mode:: "TeX:UTF-8" 
\begin{conclusions}

科技日新月异,我国综合国力在飞速提升,机器人相关技术也在迅猛发展。工业、服务智能化对机器人自适应能力提出了苛刻的要求,视觉伺服已经几乎是智能机器人不可缺少的部分。本文以手在眼机械臂视觉伺服系统为研究对象,以抓取未知形状随机位姿的物体为目标展开研究,旨在面对多样的抓取对象,与其它类似方法比较也能拥有更高的抓取成功率。在研究中提出一种基于抓取点的自定义特征,保证特征匹配的可靠性和特征排序不变性。在当前算法基础上给出适合它的IBVS控制律,针对特征偏差对末端速度指令关联的不对称性问题和噪声问题提出一种自适应控制算法。最后通过实物实验验证提出的基于抓取点生成网络和IBVS的未知物体抓取算法的高抓取率、高伺服性能。


本文主要研究内容如下:


(1)针对未知物体抓取任务搭建IBVS系统模型,其中包括系统坐标系和视觉坐标系的建立。针对以上建立的模型和基于特征点交互矩阵的IBVS原理,实现IBVS系统仿真,并实现数据曲线记录和相机轨迹绘制功能方便调试。最后搭建实物平台,在实物上成功运行IBVS算法。


(2)研究了基于模型的点云识别与配准和抓取点生成网络(GG-CNN)两种生成伺服目标的方法,通过比较它们的优劣选择了后者作为伺服目标生成方案。最后采取滤波等方式抑制了神经网络的输出波动。


(3)通过图像处理解决各种环境干扰问题。将抓取点生成网络运用到IBVS中,提出一种基于抓取点的自定义特征并根据伺服效果优化了特征分布,实现了未知物体抓取算法并在实物上成功运行。


(4)针对当前伺服性能差的问题,建立控制模型,根据伺服曲线给出IBVS控制律的改善方法。相比于原始的伺服指令曲线和特征偏差曲线,改进后的曲线更加理想且光滑。对于出现的特征偏差对末端速度指令关联的不一致性问题和噪声问题提出一种自适应控制算法。


(5)设计了抓取实验,通过多样的物体、多样的位姿的抓取实验验证提出的未知物体抓取算法的有效性。与相关工作进行比较,证明当前算法的高自适应能力和高伺服性能。


综上所述,本文对基于视觉伺服的机械臂抓取未知形体、任意位姿物体的研究取得了一定的进展,但根据当前伺服的实际情况可知在未来的研究中仍然有许多问题需要去解决。那些需要继续研究的方面将在下文列出。


(1)本文研究中使用了GG-CNN作为抓取点生成的方式之一,但在实时抓取点生成中发现抓取点位置波动严重到无法使用,所以使用了图像一阶矩代替成为获取抓取点中心位置的方案,仅使用了GG-CNN的抓取点偏转角信息。训练网络时填充自己的数据集也许能部分解决问题,但不能从根本上解决该网络对深度的平移、旋转不变性的问题。一个更好的方式配合基于抓取点的自定义特征实现IBVS,能让机械臂的自适应能力变得更好。


(2)本文提出的于抓取点的自定义特征仅能用于相机一直俯视向下的伺服系统,这对该方法的使用带来了一定的局限性。因此,如果能借助机械臂编码器信息,通过优化方法使该方法在不借助外参的情况下在六自由度速度指令的伺服系统中使用,该方法就能作为插件广泛应用于各个IBVS系统中了。


(3)尽管本文在对于视觉伺服控制律优化的研究结果是好的,但还是不够深入的。仅从时域的角度对控制律进行优化,可能会遗漏关于优化方法上的重要信息。如果能从频域角度分析指令和对象然后设计控制器可能会为这个伺服系统带来更高的伺服性能。

\end{conclusions}
